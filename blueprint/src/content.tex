\chapter{Preliminaries}
Suppose throughout that \(P\) is a poset with order relation \(\leq\).
\begin{definition}
  \label{def:directed}
  \lean{DirectedOn}
  \leanok
  We say that a subset \(S \subseteq P\) is \emph{directed} if for every \(x, y \in S\) there exists \(z \in S\), such that \(x \leq z\) and \(y \leq z\).
\end{definition}

\begin{definition}
  \label{def:lower}
  \lean{IsLowerSet}
  \leanok
  We say that \(S\) is \emph{downwards closed} if whenever \(x \in S\) and \(y \leq x\), we also have \(y \in S\).
\end{definition}

\begin{definition}
  \lean{Order.Ideal}
  \uses{def:directed, def:lower}
  \leanok
  If \(S\) is nonempty, directed and downwards closed, we call \(S\) an \emph{ideal} of \(P\).
\end{definition}

\begin{lemma}
  \lean{Order.downset_ideal}
  \leanok
  For all \(x \in P\), the set \(\downset{x} = \left\{y \in P \vert y \leq x\right\}\) is an ideal of \(P\).
\end{lemma}

\begin{lemma}
  Suppose the poset \(P\) has binary joins \(\vee\) and \(S \subseteq P\) is downwards closed.
  Then \(P\) is directed (and hence an ideal) if and only if \(P\) is closed under binary joins. 
\end{lemma}

\begin{definition}
  \lean{Order.Dcpo}
  \leanok
  Whenever a directed subset \(S \subseteq P\) has a join, we call it a \emph{directed join} and denote it by \(\dirjoin S\). If \(P\) has all directed joins, we call \(P\) a \emph{directed complete partial order}, or \emph{dcpo} for short.
\end{definition}

\begin{lemma}
  To construct arbitrary joins in a dcpo \(P\), it suffices to construct all finite joins. Furthermore, to show meets distribute over arbitrary joins, it suffices to show they distribute overy directed and finite joins.
\end{lemma}

Suppose now that \(P\) and \(Q\) are dcpos and \(f : P \to Q\) a monotone function. We define what it means for \(f\) to preserve directed joins.

\begin{lemma}
  The function \(f\) preserves directed sets. More precisely, if \(S\) is a directed set in \(P\), then \( \left\{ f(s) \vert s \in S \right\}\) is a directed set in \(Q\).
\end{lemma}

Since \(f\) is monotone, one of the inequalities always holds.

\begin{proposition}
  Suppose \(S\) is a directed set. Then
  \(\dirjoin \left\{f(s) \vert s \in S\right\} \leq f(\dirjoin S).\)
\end{proposition}

\begin{definition}
  We say that the function \(f\) is \emph{Scott continuous} if for every directed set \(S\), \(f(\dirjoin S) \leq \dirjoin \left\{f(s) \vert s \in S\right\}\)
\end{definition}

\chapter{Ideal completion}

\begin{definition}
  We define \(\idl(P)\) to be the set of all ideals of \(P\) and call it \emph{the ideal completion of \(P\)}. The set \(\idl(P)\) forms a poset under subset inclusion.
\end{definition}

\begin{proposition}
  The assignment \(x \mapsto \downset{x}\) defines a monotone function \(\eta : P \to \idl(P)\).
\end{proposition}

\begin{proposition}
  The union of a directed family of ideals is an ideal. Consequently, \(\idl(P)\) is a dcpo.
\end{proposition}

\begin{theorem}
  The poset \(\idl(P)\) forms the free dcpo over \(P\) with unit \(\eta\). Given a dcpo \(Q\) and a monotone function \(f : P \to Q\), there exists a unique Scott continuous function
  \(\overline{f} : \idl(P) \to Q\), such that \(\overline{f}(\downset{x}) = f(x)\) for all \(x \in P\).

  The function \(\overline{f}\) is defined by the assignment \(I \mapsto \dirjoin \left\{ f(x) \vert x \in I \right\}\).
\end{theorem}

\begin{corollary}
  The assignment \(P \mapsto \idl(P)\) is functorial.
\end{corollary}

Any additional joins and meets that exist in \(P\) also exist in \(\idl(P)\) and are preserved by \(\eta\).

\begin{proposition}
  Suppose \(P\) has all finite meets. Then so does \(\idl(P)\), with binary joins given by intersection and top element given by the ideal \(P\).
  Since \(\downset{(x \bmeet y)} = \downset{x} \cap \downset{y}\) and \(\downset{1} = P\), the function \(\eta\) preserves all finite meets.
\end{proposition}

\begin{proposition}
  Suppose \(P\) has all finite meets. Then meets distribute over directed joins in \(\idl(P)\).
\end{proposition}

For my project, I would like to formalize the stone duality and some related facts. We talked about the possibility of contributing to an existing formalization, but I think it would be good if I tried to plan out the project myself. I am also interested in isolating the constructive part of the proof, and possibly doing the classical part as a stretch goal. What I am planning to formalize is
1. Given a poset P, construct its ideal completion Idl(P) and show that it forms the free dcpo over P. If P is a distributive lattice, show that it forms the free frame over P, call this the coherent frame over P.
2. Introduce compact elements, show some facts about the subposet of compact elements KP. If P is a boolean algebra, show that K is functorial.
3. Show that KIdl(P) is isomorphic to P for every poset P. If P is a boolean algebra, show that this isomorphism is natural.
4. Introduce algebraic dcpos. Show that a dcpo A is algebraic if and only if it isomorphic to Idl(KA).
